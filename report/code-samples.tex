\section{Code Samples}

\subsection{Simple One-Shot Chat Server}

\begin{code}
let client = 
    out(stdout,"Enter Host"); in(stdin,host);
    out(stdout,"Enter Port"); in(stdin,port);
        let ch = chan(host,port) in
            in(stdin,msg);out(ch,msg);in(ch,msg)
\end{code}

\begin{code}
let server = 
out(stdout,"Enter Port");
in(stdin,port);
    let ch = {port} in
        in(ch,msg);in(stdin,reply);out(ch,reply)
\end{code}

\subsection{Chat Server with Anonymous Channels}
\label{sec:clientserverfancy}
Here we create an anonymous channel, send it across to a remote phi, and then communicate along it.

\begin{code}
let server = 
    out(stdout,"Enter port:"); in(stdin,port);
    let ch = {port} in 
        in(ch,sch);in(stdin,msg);out(sch,msg)
\end{code}
\begin{code}
let client = 
    out(stdout,"Enter host:"); in(stdin,host);
    out(stdout,"Enter port:"); in(stdin,port);
    let ch = chan(host,port) in
        let sch = {} in
            out(ch,sch);in(sch,msg);out(stdout,msg)
\end{code}

\subsection{Handshake Protocol}
\label{sec:handshake}

Here is an implementation of the Handshake protocol written in phi. To run it, open the REPL, \&load("handshake.pi", and you should see an endless string of s's appearing

\begin{code}
new s
let c = {9000}
let c' = chan("localhost",9000)
let clientA(pkA,skA,pkB) =  
    (out(c,pkA); in(c,x); 
    let y = adec(skA,x) in
    let pair(pkD,j) = getmsg(y) in if pkD = pkB then let k = j in
    out(c,senc(k,s)))
let serverB(pkB,skB) = 
    in(c',pkX);
    new k;
    out(c',aenc(pkX,sign(pkB,pair(pkB,k))));
    in(c',x);
    let z = sdec(k,x) in out(stdout,z)
let process = 
    (new skA;
    new skB;
    let pkA = pk(skA) in
    let pkB = pk(skB) in
    !(&clientA(pkA,skA,pkB))|!(&serverB(pkB,skB)))
\end{code}

