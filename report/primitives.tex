\section{Primitives and Term Manipulation}
\label{sec:primitives}

The Primitives module exports an associative list 
\begin{minted}{haskell}
primitives :: [(Name,TermFun)]
\end{minted}
which maps the names of primitive functions to $TermFun$ s. As explained in section \ref{subsec:types}, 
\begin{minted}{haskell}
    type TermFun = [Term] -> ThrowsError Term
\end{minted}
and
\begin{minted}{haskell}
    type ThrowsError a = Either PiError a
\end{minted}
This means each of these functions will be passed a list of terms, and then has the possibility of returning a result (a $Term$ ) or an error ( a $PiError$ )
As such, any malformed or incorrect inputs will fail gracefully and be handled appropriately.

We have implemented a large set of primitives with which you can manipulate Terms, and some specific to manipulating HTTP data. In this section we go over the primitive functions available to the user, and explain their purpose and implementation.
 
\subsection{Definitions and Explanations}

\input{prims.lhs}

Hopefully it is clear that we have built this set of primitives in such a way that allows for the user to compose them easily and build their own functions. An example of this can be found in the $pilude$: $getLocation(httpRequest)$ -- defined as $getHeader("location",httpRequest)$
