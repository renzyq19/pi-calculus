\section{Introduction}

The applied $\pi$-calculus is an expressive formal language describing computations as communicating processes. Its primary use in both industry and academia is to model security protocols. These models being built, they can then be statically analysed either by hand, or automatically by using a tool such as ProVerif. 

For the purposes of static analysis, models built using applied $\pi$-calculus have proven very useful. Applied $\pi$-calculus has been used to verify numerous security protocols, including but not limited to \cite{rs13}:
\begin{itemize}
    \item Email certification
    \item Privacy and verifiability in electronic voting
    \item Authorisation protocols in trusted computing
    \item Authentication protocols and key agreement
\end{itemize} 
However, these models are limited by the fact that they cannot currently be executed directly, as there is no existing language implementation. 

Without an implementation, any models built using the applied $\pi$-calculus cannot be used to actually demonstrate protocols they are modelling, and so those models can be very difficult to debug.

\subsection{Objective}

The aim of this project is to provide an implementation of the applied $\pi$-calculus such that one might be able to build a model of a web protocol and then execute it interoperably with existing implementations written in PHP, JavaScript or any other web language. The resulting implementation will hopefully not only be a very powerful and concise language for reasoning about and implementing protocols, but a useful scripting tool for the web.

At the very least, we would like to be able to write something similar to:
    
\begin{verbatim}
    in(a,M); 
    out(b,M);
\end{verbatim}

Compile it and have it execute successfully; receiving a message in on channel a, and sending the same message out again on channel b. However, ideally we would like to write something like the following:
% TODO
\begin{verbatim}
    out(net,httpRequest(uri,headers,httpGet()));
    in(net,(hs : list(Header), message : String));
    out(stdout, message);
    |
    in(net,(hs : list(Header), req: HttpReq));
    out(process,req);
    in(process, resp : HttpResp);
    out(net,httpResponse(origin(hs),resp));
\end{verbatim}

which would start one process which would send out an HTTP GET request on the channel net, and await a response. Once that response has been received, it will de-structure it and send the contents of the HTTP Response to stdout. Meanwhile, another process is set up to receive the same request on net, de-structure it into its component parts, then handle the request, and send back an appropriate HTTP response. The latter may well be out of the our abilities, however the we would aim to create something capable of handling something a little more impressive than the former.

\subsection{Approach}

We aim to build an interpreter for the applied $\pi$-calculus. The language we plan to do this in is Haskell, due to familiarity using both the language itself and the parsec library \cite{lm01}, a powerful parser combinator library. 

We will also need to make some restrictions as to exactly what our version of the $\pi$-calculus can do. With it in theory being such an expressive language, the scope of this exercise is potentially enormous. These restrictions will be covered in the following section.

We will also be building a little web playground for our initial efforts to interact with. This will give us a good idea of how our implementation will interact with real world servers, if at all. 



